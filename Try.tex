\documentclass[11pt]{article}
\usepackage[utf8]{inputenc}
\usepackage{indentfirst}
\usepackage[a4paper, total={6in, 8in}, margin=1.2in, bottom=1in	]{geometry}
\usepackage{titling}
\usepackage{amsmath}
\usepackage{enumitem}
\usepackage{fixltx2e}
\usepackage{bera}
\usepackage{multirow}
\usepackage{graphicx}
\usepackage{hhline}
\usepackage{amssymb}

\begin{document}

\begin{titlingpage}
\pagenumbering{gobble}
\title{\textbf{Software Systems Lab: OutLab 5\\\LaTeX{} (80 Marks)}}
\author{\Large Shinigami\\\\\normalsize 180050007\\\normalsize 180050011\\\normalsize 180050075}
\date{15 September, 2019}
\end{titlingpage}

\maketitle
\clearpage
\pagenumbering{arabic}
\newpage
\tableofcontents
\newpage
This is a \LaTeX{} document for the course {\textbf{Software Systems Lab}} with course code {\textit{CS 251}}. You need to replicate the document. Spacing need not be macthed perfectly but page numbers should be. {\textbf{1 mark is for the title page.}} Bonus marks will be given only if you score full (80/80) in the rest.

\section{Introduction (4 marks)}
{\textbf{\LaTeX{}}} is a word processor and document markup language. It is distinguished from typical word processors such as Microsoft Word and Apple Pages in that the writer uses plain text as opposed to formatted text, relying on markup tagging conventions to define the general structure of a  document (such as article, book, and letter), to stylise text throughout a document (such as {\textbf{bold}} and italic), and to add citations and cross-referencing. A {\textbf{\TeX{}}} distribution such as {\textbf{\TeX{}} Live} or {\textbf{MikTeX}} is used to produce an output file (such as PDF or DVI) suitable for printing or digital distribution.\\

{\textbf{\LaTeX{}}} is used for the communication and publication of scientific documents in many fields, including mathematics, physics, computer science, statistics, economics, and political science. It also has a prominent role in the preparation and publication of books and articles that contain complex multilingual materials, such as Sanskrit and Arabic. {\textbf{\LaTeX{}}} uses the TeX typesetting program for formatting its output, and is itself written in the TeX macro language.\\

{\textbf{\LaTeX{}}} is widely used in academia. {\textbf{\LaTeX{}}} can be used as a standalone document preparation system, or as an intermediate format. In the latter role, for example, it is often used as part of a pipeline for translating DocBook and other XML-based formats to PDF. The typesetting system offers programmable desktop publishing features and extensive facilities for automating most aspects of typesetting and desktop publishing, including numbering and cross-referencing of tables and figures, chapter and section headings, the inclusion of graphics, page layout, indexing and bibliographies.\\

Below are some of the basic packages which you’ll be using. For other required packages, search over the net :).

\subsection{graphicx package}
\noindent 
This package is used to import tables, and figure in the document. Our document type is article, and we are currently using a4 type paper with the following margin geometry: (total={6in, 8in}, margin=1.2in, bottom=1in), which is specified in the beginning.

\subsection{amssymb package}
\noindent
This package is used to import mathematical symbols in the document. We
encapsulate the mathematical equations and symbols under \$, and they are
changed to maths symbols.

\section{Pointers (3 + 2 + 1 marks)}
\noindent
Here we are using {\textbf{itemize}} to generate unordered list.
\begin{itemize}
	\item \LaTeX{} typesets a file of text using the TEX program.\\
	\item \LaTeX{} is widely used in academia for the communication and publication of scientific documents in many fields, including mathematics, physics, computer science, statistics, economics and political science.\\
	\item {\textbf{We have used renewcommand for the bullets to be bigger.}}\\
	\item Look at the {\textbf{item separation space}}, and {\textbf{change it}} accordingly.	
\end{itemize}

\noindent For ordered lists we use {\textbf{enumerate}}.

\begin{enumerate}[label=\Roman*]
	\item {\LaTeX{}} typesets a file of text using the TEX program.
	\item {\LaTeX{}} is widely used in academia for the communication and publication of scientific documents in many fields, including mathematics, physics, computer science, statistics, economics and political science.
	\item {\LaTeX{}} can be used as a standalone document preparation system or as an intermediate format.
	\item {\LaTeX{}} is intended to provide a high-level language that accesses the power of TeX in an easier way for writers.
\end{enumerate}

\begin{enumerate}[label=(\alph*)]
	\item {\LaTeX{}} typesets a file of text using the TEX program.
	\item {\LaTeX{}} is widely used in academia for the communication and publication of scientific documents in many fields, including mathematics, physics, computer science, statistics, economics and political science.\\
\end{enumerate}

Following is another type of a pointer ({\textbf{description}}).\\
\begin{description}
    \item[CS 213] Data Structures and Algorithm
    \item[CS 215] Data Analysis and Interpretation
    \item[CS 251] Software Systems Lab
\end{description}
\newpage

\section{Mathematical formulae and notations (15 marks)}

\subsection{Equation Array (4 marks)}
\begin{eqnarray}
	\cos^3 \theta + \sin^3 \theta & = & (\cos \theta + \sin \theta)(\cos 2\theta - \cos \theta \sin \theta) \\
            	 				  & = & (\cos \theta + \sin \theta)(1 - \cos \theta \sin \theta) \\
            	 				  & = & (\cos \theta + \sin \theta)(1/2)(2 - 2\cos \theta \sin \theta)(3) \\
            	 				  & = & (1/2)(\cos \theta + \sin \theta)(2 - \sin (2\theta))
\end{eqnarray}

\subsection{Prepositional Formulae using Various Operators (2 marks)}
$ (\exists x)(\varphi (x) \wedge \psi (x)) \longleftrightarrow ((\exists x) \varphi (x) \wedge (\exists x) \psi (x)) $

$ (\exists x)(\varphi (x) \wedge \psi (x)) \longrightarrow ((\exists x)\psi (x) \wedge (\exists x) \psi (x) \wedge (\exists x)\varphi (x)) $

\subsection{Alphabets (3 marks + 1 mark for table)}
%$ \times \otimes \oplus \cup \cap $
%$ \subset \supset \subseteq \supseteq \less \gtr $
\begin{center}
\begin{tabular} { |c|c| }
 \hline
  Binary Operators: & $ \times \otimes \oplus \cup \cap $\\
  & \\
 \hline
  Relation Operators: & $ \subset \ \supset \ \subseteq \ \supseteq \ \textbf{< >} $ \\
  & \\
 \hline
  Others: & $ \int \ \oint \ \sum \ \prod $ \\
  & \\
 \hline
\end{tabular}
\end{center}

\subsection{Mathematical Formulas (5 marks)}
\renewcommand{\theenumi}{\arabic{enumi}}
\begin{enumerate}
	\item
	$
	\!
	\begin{aligned}[t]
 	\int_a^b x^3 dx
    	&= \frac{1}{4} x^4 \biggr|_a^b &&
	\end{aligned}
	$
	\item
	$
	\!
	\begin{aligned}[t]
 	\frac{\pi}{4}
    	&= 4 \sum_{n=0}^{\infty} \frac{(-1)^n}{(2n+1)5^{2n+1}} - \sum_{n=0}^{\infty} \frac{(-1)^n}{(2n+1)239^{2n+1}} &&set 
	\end{aligned}
	$
	\item
	$
	\!
	\begin{aligned}[t]
 	\pi
    	&= \frac{3\sqrt{3}}{4} - 24\sum_{n=0}^{\infty} \frac{\frac{(2n)!}{(n)}}{2n+1(2n+1)4^{2n+1}} &&
	\end{aligned}
	$
	\item
	$
	\!
	\begin{aligned}[t]
 	\frac{1}{\pi}
    	&= \frac{2\sqrt{2}}{9801} \sum_{n=0}^{\infty} \frac{(4n)!(1103+26390n)}{(n)!^4 396^{4n}} &&
	\end{aligned}
	$
	\item
	$
	\!
	\begin{aligned}[t]
 	\sum\nolimits_{i=0}^{[\frac{n}{2}]} {{x_{i,i+1}^{i^2}} \choose {[\frac{i+3}{3}]}} {\frac{\sqrt{{\mu (i)}^{\frac{3}{2}} (i^{2}-1)}}{\sqrt[3]{\rho (i)-2} + \sqrt[3]{\rho (i)-1}}}
	\end{aligned}
	$
\end{enumerate}

\newpage
\section{Tables (10 marks)}
\noindent
To combine rows a package must be imported with in your preamble, then you
can use the XXXXXXX command in your document. The table below includes
mathematical notations, which you can produce by embedding the expression
in \$ \$ delimiters. For subscript, use underscore and for superscript, use carrot.

\begin{center}
\begin{tabular}{ |c|c|c|c|c|c|c|c|c|c|c|c| }
\hline
\multirow{2}{5em}{\textit{Baseline}} & Mean & 0.84 & 0.41 & \textbf{0.56} & \textbf{0.46} & \textbf{0.55} & \textbf{0.60} & 0.56 & 0.57 & 0.63 \\ 
\hhline{||~|-|-|-|-|-|-|-|-|-|-|-||}
& SD & 0.07 & 0.08 & 0.06 & 0.07 & 0.05 & 0.05 & 0.06 & 0.07 & 0.05 \\  
\hline
\hline
\multirow{2}{5em}{\textit{ScaComp\textsubscript{h}}} & Mean & 0.89 & 0.46 & 0.53 & 0.43 & 0.53 & 0.58 & 0.54 & 0.56 & 0.62 \\ 
\hhline{||~|-|-|-|-|-|-|-|-|-|-|-||}
& SD & 0.05 & 0.08 & 0.05 & 0.06 & 0.06 & 0.05 & 0.05 & 0.06 & 0.05 \\  
\hline
\hline
\multirow{2}{5em}{\textit{ScaComp\textsubscript{l}}} & Mean & \textbf{0.92} & \textbf{0.48}& 0.55 & 0.45 & 0.53 & 0.59 & \textbf{0.58} & \textbf{0.61} & \textbf{0.64} \\ 
\hhline{||~|-|-|-|-|-|-|-|-|-|-|-||}
& SD & 0.04 & 0.07 & 0.05 & 0.04 & 0.05 & 0.04 & 0.04 & 0.04 & 0.04 \\  
\hline
\end{tabular}
\end{center}

\end{document}

